\documentclass[10pt]{article}
\usepackage[utf8]{inputenc}

% Page setup
\usepackage[a3paper,landscape,margin=2cm]{geometry}

% Typography
\usepackage[T1]{fontenc}
\usepackage[scaled]{berasans}
\usepackage[scaled]{beramono}
\renewcommand*\familydefault{\sfdefault}
\usepackage{microtype}
\parindent 0pt

% TikZ
\usepackage{./tikz-uml}
\usetikzlibrary{positioning}

% Headings
\makeatletter
\def\@maketitle{%
  {\LARGE\bf\@title\par}
  \vskip .5em
  {\large\@author\ -- \@date\par}
  \vskip 2em
}
\makeatother

% Notes
\usepackage{multicol}
\newenvironment{Note}
  {\begin{multicols}{3}%
     \parskip 1em}
  {\end{multicols}}

% Document metadata
\title{
  Solid server -- Proposed architecture v1.1.0
  \it (status: draft)
}
\author{Ruben Verborgh}

\begin{document}

\maketitle

\section*{Purpose}
This document conveys a~personal view
on important architectural considerations for a~Solid server.

It is intended as a~tool for discussion,
to raise questions,
and to highlight concerns.

It does not have any official standing whatsoever.

\newcommand\ResourceStoreBody{%
  + getRepresentation(ResourceIdentifier, RepresentationPreferences) : Promise<Representation>\\
  + addResource(container : ResourceIdentifier, Representation) : Promise<ResourceIdentifier>\\
  + setRepresentation(ResourceIdentifier, Representation) : Promise<void>\\
  + deleteResource(ResourceIdentifier) : Promise<void>\\
  + modifyResource(ResourceIdentifier, Patch) : Promise<void>\\
}

\section*{Overview of LDP and Access Control}
\begin{tikzpicture}

\begin{umlpackage}{HTTP}
  \umlinterface[]{HttpHandler}{}{
    + canHandle(HttpRequest) : Promise<boolean>\\
    + handle(HttpRequest, HttpResponse) : void\\
  }

  \umlclass[right=1 of HttpHandler]{HttpServer}{}{
    + HttpServer(Array<HttpHandler>)\\
    + listen(port : int) : void\\
    + handle(HttpRequest, HttpResponse) : void\\
  }
  \umluniaggreg[mult=*]{HttpServer}{HttpHandler}
\end{umlpackage}

\begin{umlpackage}[y=-4.5]{LDP}
  \umlclass{LdpHandler}{}{
    + LdpHandler(OperationFactory)\\
  }
  \umlimpl{LdpHandler}{HttpHandler}


  \umlsimpleclass[right=2 of LdpHandler.north east,anchor=south west]{TargetExtractor}
  \umluniaggreg[mult=1,anchor2=west]{LdpHandler}{TargetExtractor}

  \umlsimpleclass[right=2 of TargetExtractor]{ResourceIdentifier}
  \umldep[arg=creates,pos=0.5]{TargetExtractor}{ResourceIdentifier}


  \umlsimpleclass[below=0.25 of TargetExtractor.south east,anchor=north east]{BodyParser}
  \umluniaggreg[mult=*,anchor2=west]{LdpHandler}{BodyParser}

  \umlsimpleclass[below=0.25 of ResourceIdentifier.south west,anchor=north west]{Representation}
  \umldep[arg=creates,pos=0.5]{BodyParser}{Representation}

  \umlsimpleclass[right=1 of Representation]{Patch}
  \umlinherit{Patch}{Representation}


  \umlsimpleclass[below=0.25 of BodyParser.south east,anchor=north east]{PreferenceParser}
  \umluniaggreg[mult=1,anchor2=west]{LdpHandler}{PreferenceParser}

  \umlsimpleclass[below=0.25 of Representation.south west,anchor=north west]{RepresentationPreferences}
  \umldep[arg=creates,pos=0.5]{PreferenceParser}{RepresentationPreferences}


  \umlsimpleclass[below=0.25 of PreferenceParser.south east,anchor=north east]{ResponseWriter}
  \umluniaggreg[mult=1,anchor1=south east,anchor2=west]{LdpHandler}{ResponseWriter}

  \umlsimpleclass[below=0.25 of RepresentationPreferences.south west,anchor=north west]{HttpResponse}
  \umldep[arg=writes,pos=0.5]{ResponseWriter}{HttpResponse}


  \begin{umlpackage}[x=6,y=-4]{Operations}
    \umlclass[]{OperationFactory}{}{
      + OperationFactory(ResourceStore)\\
      + createOperation(method : string, ResourceIdentifier,\\*
                        RepresentationPreferences, \ldots) : Operation\\
    }
    \umluniaggreg[mult=1,geometry=|-]{LdpHandler}{OperationFactory}

    \umlinterface[below=1 of OperationFactory]{Operation}{
      + target : ResourceIdentifier\\
      + requestBody : Representation?\\
      + preferences : RepresentationPreferences\\
      + requiredPermissions : PermissionSet\\
    }{
      + execute() : Promise<ResponseDescription>\\
    }
    \umldep[arg=creates,pos=0.5]{OperationFactory}{Operation}

    \umlsimpleclass[right=1 of Operation.north east,anchor=north west]{GetOperation}
    \umlsimpleclass[below=.25 of GetOperation.south west,anchor=north west]{PostOperation}
    \umlsimpleclass[below=.25 of PostOperation.south west,anchor=north west]{PutOperation}
    \umlsimpleclass[below=.25 of PutOperation.south west,anchor=north west]{PatchOperation}
    \umlimpl[anchor1=west]{GetOperation}{Operation}
    \umlimpl[anchor1=west]{PostOperation}{Operation}
    \umlimpl[anchor1=west]{PutOperation}{Operation}
    \umlimpl[anchor1=west]{PatchOperation}{Operation}

  \end{umlpackage}
\end{umlpackage}

\begin{umlpackage}{Storage}
  \umlinterface[below=2.25 of Operation]{ResourceStore}{}{
    \ResourceStoreBody
  }
  \umluniaggreg[mult=1]{Operation}{ResourceStore}
\end{umlpackage}

\begin{umlpackage}{Authentication}
  \umlinterface[left=2 of LdpHandler.west]{CredentialsExtractor}{}{
    + extractCredentials(HttpRequest) : Promise<Credentials>\\
  }
  \umluniaggreg[mult=1]{LdpHandler}{CredentialsExtractor}

  \umlsimpleclass[below=1 of CredentialsExtractor]{Credentials}
  \umldep[arg=creates,pos=0.5]{CredentialsExtractor}{Credentials}
\end{umlpackage}

\begin{umlpackage}{Authorization}
  \umlinterface[below=4.5 of CredentialsExtractor]{Authorizer}{}{
    + hasPermissions(Credentials,
                     ResourceIdentifier,\\*
                     PermissionSet) : Promise<boolean>\\
  }
  \umldep[arg=uses,pos=0.3]{Authorizer}{Credentials}
  \umluniaggreg[mult=1,geometry=|-,weight=0.6]{LdpHandler}{Authorizer}

  \umlsimpleclass[below=1 of Authorizer]{AclBasedAuthorizer}
  \umlimpl{AclBasedAuthorizer}{Authorizer}
  \umluniaggreg[mult=1,geometry=|-]{AclBasedAuthorizer}{ResourceStore}
\end{umlpackage}

\end{tikzpicture}

\clearpage

\section*{Resources and Representations}
\begin{Note}
The intention of \textbf{ResourceIdentifier} and \textbf{Representation}
is to capture the REST notion of a~resource and its representation.
In~the~case of a~photograph,
the resource is the photograph itself,
whereas a~representation is a~concrete manifestation of that photograph
with a~certain resolution and file~type.
In~the~case of an RDF document,
the~resource is the RDF graph,
and concrete representations serialize that graph
into Turtle or specific framings of~JSON-LD.
\columnbreak

For all practical purposes,
\textbf{ResourceIdentifier} can just be a~\textbf{URL};
the terminology is mainly used to emphasize
the resource/representation notion of REST.
Also, there is no \textbf{Resource} class,
because resources are always manipulated through representations in REST,
so we only need to \emph{identify} resources,
and only deal with them through their representations.
\columnbreak

Crucially, as the diagram below shows,
the \textbf{Representation} interface
can have vastly different underlying in-memory structures,
such as strings, binary streams, RDF streams, etc.
So they can be photographs as well as RDF streams,
and most other classes handling them do not need to care.
This enables back-ends to be RDF-aware when they need to,
and RDF-oblivious when they do~not.
\end{Note}

\bigskip

\begin{tikzpicture}
  \umlinterface{Representation}{
    + identifier : ResourceIdentifier?\\
    + metadata : RepresentationMetadata\\
    + data : Readable<Object>\\
    + dataType : String\\
  }{}

  \umlinterface[left=2 of Representation]{ResourceIdentifier}{
  }{}
  \umluniaggreg[mult=0..1]{Representation}{ResourceIdentifier}

  \umlinterface[below=2 of Representation]{RepresentationMetadata}{
    + byteSize : int?\\
    + contentType : String?\\
    + encoding : String?\\
    + language : String?\\
    + dateTime : Date?\\
    + profiles : Array<String>\\
  }{}
  \umluniaggreg[mult=1]{Representation}{RepresentationMetadata}

  \umlclass[right=2 of Representation.north east]{BinaryRepresentation}{
    + data : Readable<Buffer>\\
  }{}
  \umlimpl{BinaryRepresentation}{Representation}

  \umlclass[right=2 of Representation.south east]{QuadRepresentation}{
    + data : Readable<Quad>\\
  }{}
  \umlimpl{QuadRepresentation}{Representation}

\end{tikzpicture}

\bigskip

\begin{Note}
The \textbf{dataType} field returns the name of the class
that elements of the \textbf{data} readable stream will have,
for instance,
\verb!Buffer! or \verb!Quad!.
\columnbreak

Based on the \textbf{dataType} and \textbf{metadata} fields,
other components can decide whether or not the representation
is acceptable to the user agent,
and, if this is not the case,
convert to a~format that is.
For instance,
a~\verb!text/turtle! stream is acceptable
for a~user agent that requested \verb!text/*!,
whereas a~\verb!Readable<Quad>! will still require serialization.

\columnbreak

\null
\end{Note}

\clearpage

\section*{ResourceStore}
\begin{tikzpicture}
  \umlinterface{ResourceStore}{}{\ResourceStoreBody}


  % FileSystemStore

  \umlclass[right=2 of ResourceStore]{FileSystemStore}{
    - ResourceMapper mapper\\
  }{}
  \umlimpl{FileSystemStore}{ResourceStore}

  \umlinterface[right=1 of FileSystemStore]{ResourceMapper}{}{
    + mapFilePathToUrl(File): Promise<\{URL, RepresentationMetadata\}>\\
    + mapUrlToFilePath(URL, RepresentationMetadata): Promise<File>\\
  }
  \umluniaggreg[mult=1]{FileSystemStore}{ResourceMapper}


  % KeyValueStore

  \umlabstract[above=1 of FileSystemStore.north west,anchor=south west]{KeyValueStore}{}{
    + get(ResourceIdentifier) : Promise<BinaryRepresentation>\\
    + replace(ResourceIdentifier, BinaryRepresentation) : Promise<void>\\
    + delete(ResourceIdentifier) : Promise<void>\\
  }
  \umlimpl[anchor1=west,anchor2=north east]{KeyValueStore}{ResourceStore}

  \umlsimpleclass[right=1 of KeyValueStore.north east]{RedisStore}
  \umlinherit{RedisStore}{KeyValueStore}

  \umlsimpleclass[right=1 of KeyValueStore.east]{CassandraStore}
  \umlinherit{CassandraStore}{KeyValueStore}


  % TripleStore

  \umlclass[below=1 of FileSystemStore.south west,anchor=north west]{TripleStore}{}{
    + getRepresentation(\ldots) : Promise<QuadRepresentation>\\
  }
  \umlimpl[anchor1=west,anchor2=south east]{TripleStore}{ResourceStore}

  \umlclass[right=1 of TripleStore]{SparqlEndpointStore}{
    - endpoint: URL\\
  }{}
  \umlinherit{SparqlEndpointStore}{TripleStore}


  % RepresentationConvertor

  \umlclass[below=1 of TripleStore.south west,anchor=north west]{RepresentationAdaptor}{
    - source : ResourceStore\\
    - convertors : Array<RepresentationConvertor>\\
  }{}
  \umlimpl[anchor1=west,anchor2=-30,geometry=-|]{RepresentationAdaptor}{ResourceStore}

  \umlinterface[right=1 of RepresentationAdaptor]{RepresentationConvertor}{}{
    + supports(Representation, RepresentationPreferences): Promise<boolean>\\
    + convert(Representation, RepresentationPreferences): Promise<Representation>\\
  }
  \umluniaggreg[mult=*]{RepresentationAdaptor}{RepresentationConvertor}


  % CompositeResourceStore

  \umlclass[below=1 of RepresentationAdaptor.south west,anchor=north west]{CompositeResourceStore}{
    - sources : Map<UrlPattern, ResouceStore>\\
  }{}
  \umlimpl[anchor1=173,anchor2=-40,geometry=-|]{CompositeResourceStore}{ResourceStore}
  \umlunicompo[mult=*,anchor1=187,anchor2=-60,geometry=-|]{CompositeResourceStore}{ResourceStore}

\end{tikzpicture}

\bigskip

\begin{Note}
A~\textbf{ResourceStore} will \emph{try} to satisfy
any \textbf{RepresentationPreferences} passed to it,
but only if this is reasonably easy for the store in question.
For instance, a~SPARQL endpoint can typically
generate N-Triples as easily as Turtle,
so it makes sense to directly generate N-Triples
if the client prefers this.
On the other hand,
a~file system will typically only have one representation on disk,
so it is fine to always serve that single representation,
regardless of client preferences.
\columnbreak

The \textbf{RepresentationAdaptor} \emph{could} be used
to satisfy client preferences more accurately.
It has access to multiple \textbf{RepresentationConvertor} instances,
which could (for instance) convert a~stream of quads
into Turtle or a~specific JSON-LD frame.
It can decorate any existing \textbf{ResourceStore}
to extend it with more kinds of representations
such as different content types.
\columnbreak

A~\textbf{CompositeResourceStore} can be used
to have multiple back-ends on one pod,
each answering to different URL patterns.
This mechanism \emph{could} be used
also to serve large files like images,
or static assets such as apps or scripts.

\end{Note}

\end{document}
