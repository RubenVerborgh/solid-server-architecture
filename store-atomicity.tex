\documentclass[10pt]{article}
\usepackage[utf8]{inputenc}

% Page setup
\usepackage[a3paper,landscape,margin=2cm]{geometry}

% Typography
\usepackage[T1]{fontenc}
\usepackage[scaled]{berasans}
\usepackage[scaled]{beramono}
\renewcommand*\familydefault{\sfdefault}
\usepackage{microtype}
\usepackage{fnpct}
\parindent 0pt

% TikZ
\usepackage{./tikz-uml}
\usetikzlibrary{positioning}

% Headings
\makeatletter
\def\@maketitle{%
  {\LARGE\bf\@title\par}
  \vskip .5em
  {\large\@author\ -- \@date\par}
  \vskip 2em
}
\makeatother

% Notes
\usepackage{multicol}
\newenvironment{Note}
  {\begin{multicols}{3}%
     \parskip 1em}
  {\end{multicols}}

% Document metadata
\title{
  Solid server -- Store atomicity
  \it (status: draft)
}
\author{Ruben Verborgh}

\begin{document}

\maketitle

\section*{ResourceStore and atomic operations}
\begin{tikzpicture}
  \umlinterface{ResourceStore}{}{
  + getRepresentation(ResourceIdentifier, RepresentationPreferences) : Promise<Representation>\\
  + addResource(container : ResourceIdentifier, Representation) : Promise<ResourceIdentifier>\\
  + setRepresentation(ResourceIdentifier, Representation) : Promise<void>\\
  + deleteResource(ResourceIdentifier) : Promise<void>\\
  + modifyResource(ResourceIdentifier, Patch) : Promise<void>\\
  }

  \umlinterface[right=2 of ResourceStore.north east, anchor=south west]{AtomicResourceStore}{}{
  }
  \umlimpl{AtomicResourceStore}{ResourceStore}

  \umlclass[right=2 of ResourceStore.east]{LockingResourceStore}{
    - ResourceStore source\\
    - ResourceLocker locks\\
  }{
  }
  \umlimpl{LockingResourceStore}{ResourceStore}
  \umlimpl{LockingResourceStore}{AtomicResourceStore}
  \umlunicompo[mult=1,anchor1=197,anchor2=-5]{LockingResourceStore}{ResourceStore}

  \umlinterface[right=2 of LockingResourceStore.east]{ResourceLocker}{
  }{
    + acquire(ResourceIdentifier) : Promise<Lock>\\
  }
  \umluniaggreg[mult=1]{LockingResourceStore}{ResourceLocker}

  \umlinterface[below=1 of ResourceLocker.south]{Lock}{
  }{
    + release() : Promise<void>\\
  }
  \umldep[arg=creates,pos=0.5]{ResourceLocker}{Lock}

\end{tikzpicture}

\hyphenation{ResourceStore}

\begin{Note}
The \textbf{ResourceStore} interface has been designed
such that each of its methods can be implemented
in an \emph{atomic} way:
for each CRUD operation%
\footnote{%
  There are 5~operations rather than~4
  because we distinguish between
  full representations update for \texttt{PUT}
  and partial updates for \texttt{PATCH}.
},
only one dedicated method needs to be called.
It is up to the implementer of the interface
to (not) make an implementation atomic.
For some implementations,
such as triple stores or other database back-ends,
atomicity is a~given.
We~\emph{could} explicitly indicate atomicity
by having such implementations
implement the (otherwise empty) \textbf{AtomicResourceStore} interface
as a~tag.

\columnbreak

Some implementations are \emph{not} atomic by default,
such as a~file system,
where a~read+append sequence could unknowingly
be interrupted by a~write
that thereby breaks atomicity.
Such non-atomic stores could be made atomic
by decorating them with a~\textbf{LockingResourceStore}.
This class wraps another \textbf{ResourceStore}
with a~locking mechanism,
which can be implemented in different ways.
An example method implementation is listed on the right.

\columnbreak

\begin{verbatim}
async function modifyResource(id, patch) {
  const lock = await this._locks.acquire(id);
  try { return await this._source.modifyResource(id, patch); }
  finally { await lock.release(); }
}
\end{verbatim}

\end{Note}

\end{document}
