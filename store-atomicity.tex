\documentclass[10pt]{article}
\usepackage[utf8]{inputenc}

% Page setup
\usepackage[a3paper,landscape,margin=2cm]{geometry}

% Typography
\usepackage[T1]{fontenc}
\usepackage[scaled]{berasans}
\usepackage[scaled]{beramono}
\renewcommand*\familydefault{\sfdefault}
\usepackage{microtype}
\usepackage{fnpct}
\parindent 0pt

% TikZ
\usepackage{./tikz-uml}
\usetikzlibrary{positioning}

% Headings
\makeatletter
\def\@maketitle{%
  {\LARGE\bf\@title\par}
  \vskip .5em
  {\large\@author\ -- \@date\par}
  \vskip 2em
}
\makeatother

% Notes
\usepackage{multicol}
\newenvironment{Note}
  {\begin{multicols}{3}%
     \parskip 1em}
  {\end{multicols}}

% Lists
\usepackage{enumitem}
\setlist{nosep,topsep=-10pt}

% Document metadata
\title{
  Solid server -- Store atomicity
  \it (status: draft)
}
\author{Ruben Verborgh}

\begin{document}

\maketitle

\section*{ResourceStore and atomic operations}
\begin{tikzpicture}
  \umlinterface{ResourceStore}{}{
  + getRepresentation(ResourceIdentifier, RepresentationPreferences) : Promise<Representation>\\
  + addResource(container : ResourceIdentifier, Representation) : Promise<ResourceIdentifier>\\
  + setRepresentation(ResourceIdentifier, Representation) : Promise<void>\\
  + deleteResource(ResourceIdentifier) : Promise<void>\\
  + modifyResource(ResourceIdentifier, Patch) : Promise<void>\\
  }

  \umlinterface[right=2 of ResourceStore.north east, anchor=south west]{AtomicResourceStore}{}{
  }
  \umlimpl{AtomicResourceStore}{ResourceStore}

  \umlclass[right=2 of ResourceStore.east]{LockingResourceStore}{
    - ResourceStore source\\
    - ResourceLocker locks\\
  }{
  }
  \umlimpl{LockingResourceStore}{ResourceStore}
  \umlimpl{LockingResourceStore}{AtomicResourceStore}
  \umlunicompo[mult=1,anchor1=197,anchor2=-5]{LockingResourceStore}{ResourceStore}

  \umlinterface[right=2 of LockingResourceStore.east]{ResourceLocker}{
  }{
    + acquire(ResourceIdentifier) : Promise<Lock>\\
  }
  \umluniaggreg[mult=1]{LockingResourceStore}{ResourceLocker}

  \umlinterface[below=1 of ResourceLocker.south]{Lock}{
  }{
    + release() : Promise<void>\\
  }
  \umldep[arg=creates,pos=0.5]{ResourceLocker}{Lock}

\end{tikzpicture}

\hyphenation{ResourceStore}

\begin{Note}
The \textbf{ResourceStore} interface has been designed
such that each of its methods can be implemented
in an \emph{atomic} way:
for each CRUD operation%
\footnote{%
  There are 5~operations rather than~4
  because we distinguish between
  full representations update for \texttt{PUT}
  and partial updates for \texttt{PATCH}.
},
only one dedicated method needs to be called.
It is up to the implementer of the interface
to (not) make an implementation atomic.
For some implementations,
such as triple stores or other database back-ends,
atomicity is a~given.
We~\emph{could} explicitly indicate atomicity
by having such implementations
implement the (otherwise empty) \textbf{AtomicResourceStore} interface
as a~tag.

\columnbreak

Some implementations are \emph{not} atomic by default,
such as a~file system,
where a~read+append sequence could unknowingly
be interrupted by a~write
that thereby breaks atomicity.
Such non-atomic stores could be made atomic
by decorating them with a~\textbf{LockingResourceStore}.
This class wraps another \textbf{ResourceStore}
with a~locking mechanism,
which can be implemented in different ways.
An example method implementation is listed on the right.

\columnbreak

\begin{verbatim}
async function modifyResource(id, patch) {
  const lock = await this._locks.acquire(id);
  try { return await this._source.modifyResource(id, patch); }
  finally { await lock.release(); }
}
\end{verbatim}

\end{Note}

\subsection*{Design considerations}

\begin{Note}
It is important to emphasize
that atomicity is \emph{not} the only reason
for the design of the \textbf{ResourceStore} interface.
The other consideration is in the 5\textsuperscript{th} method
\verb!modifyResource!,
which allows us to optimize modifications in a~backend-specific way.
Since we expect small modifications to larger resources
to be a~common pattern for Solid apps,
we need to be able to handle those efficiently.

\columnbreak

A~simpler implementation with 4~methods
could support \verb!PATCH! as follows:
\begin{enumerate}
  \item call \verb!getRepresentation!
  \item apply the patch
  \item call \verb!setRepresentation!
\end{enumerate}

However, in addition to violating atomicity
(or requiring another locking mechanism),
it would also give suboptimal results
when the resource is large
and the patch is just a~single triple.
Moreover, it would be unnecessarily complex and slow
for the case of triple stores,
which support patches natively.

\columnbreak
In contrast,
\verb!modifyResource! gives implementations the freedom
on how to apply patches,
such that they can pick whichever option is most efficient
for a~given patch
and, if desired, support atomicity.

\end{Note}

\section*{ResourceStore and conditional requests}
\begin{Note}
With the above,
we have established that \textbf{ResourceStore}:
\begin{itemize}
  \item supports all CRUD requests;
  \item can support all types of patches efficiently;
  \item support atomicity
        (regardless of native support by the back-end).
\end{itemize}

\columnbreak
However, the proposed mechanism
does \emph{not} support conditional HTTP requests (RFC~7232),
which must be aborted if the resource
prior to modification
does not satisfy certain conditions.
These are not supported because:
\begin{itemize}
  \item \textbf{ResourceStore} cannot abort,
        because it does not know the conditions.
  \item Callers of \textbf{ResourceStore} know the conditions,
        but they cannot check them in an atomic way,
        since they would not be able to prevent modifications
        in between the \verb!getRepresentation! call for checking the conditions,
        and the subsequent modification call.
\end{itemize}

\columnbreak
Hence,
we explore three different extensions to the architecture
that aim to support conditional requests,
and analyze their properties.

\end{Note}

\clearpage

\section*{Approach 1 to conditional requests: \emph{transactions}}
\begin{tikzpicture}
  \umlinterface{ResourceStore}{}{
  + createTransaction(ResourceIdentifier) : Promise<ResourceTransaction>\\
  }

  \umlinterface[right=2 of ResourceStore]{ResourceTransaction}{}{
  + getRepresentation(RepresentationPreferences) : Promise<Representation>\\
  + addResource(Representation) : Promise<ResourceIdentifier>\\
  + setRepresentation(Representation) : Promise<void>\\
  + deleteResource() : Promise<void>\\
  + modifyResource(Patch) : Promise<void>\\
  + release() : Promise<void>\\
  }
  \umldep[arg=creates,pos=0.5]{ResourceStore}{ResourceTransaction}
\end{tikzpicture}

\subsection*{Description}

\begin{Note}
The original \textbf{ResourceStore} methods
are moved into a~\textbf{ResourceTransaction} interface,
which gets an additional \verb!release! method
to end the transaction.
The caller becomes responsible for steering the atomicity
(but the implementation remains with the \textbf{ResourceStore}).

\columnbreak
Implementations do not need to be
(and likely would not be)
\emph{actual} transactions,
in the sense that operations do not \emph{need}
to be buffered until the very end when \verb!release! is called.
They rather can function as \emph{locks/semaphores}
that guarantee no other operations
can happen in the meantime.

\columnbreak
\null

\end{Note}

\subsection*{Analysis}

\begin{Note}
Requirements:
\begin{itemize}
  \item The caller knows how to validate request conditions.

  \item Every \textbf{ResourceStore} implementation must be transaction-aware
        (or at least \emph{lock}-aware),
        which is not the case for back-ends such as files.

  \item Callers of \textbf{ResourceStore} must be transaction-aware.
\end{itemize}

\bigskip
When a~conditional request arrives, implementers must:
\begin{enumerate}
  \item call \verb!createTransaction!
  \item call \verb!getRepresentation!
  \item check the conditions
  \item if the conditions are satisfied, call the modification method
  \item call \verb!release!
\end{enumerate}

\bigskip
This comes with a~couple of caveats:
\begin{itemize}
  \item We probably do not want to retrieve the full representation,
        but only the metadata (lazy loading can do that).

  \item It might result in the representation (or its metadata)
        being loaded twice:
        once by the caller when getting the representation,
        and once by the store internally when performing the modification.
        (The transaction can, however, cache this.)

  \item It assumes that \verb!getRepresentation! succeeds,
        which might not be the case for append-only stores.

  \item It assumes that \verb!createTransaction! is sufficiently cheap.

  \item It assumes that keeping a~lock/transaction open is sufficiently cheap.

  \item In general,
        it assumes that the caller has the best knowledge
        for checking the conditions in the cheapest way possible,
        which is not necessarily true.
        For instance,
        for one store it might be expensive to calculate ETag
        but not the last modified date,
        whereas it might be the opposite for another.
        A~certain store might even be able to determine
        that a~condition is met \emph{without} retrieving
        a~representation or its metadata
        (for instance, if its global last-modified date
         is not later than the requested one).
\end{itemize}

\columnbreak
\null


\columnbreak
\null

\end{Note}


\section*{Approach 2 to conditional requests: \emph{passing a~validator}}
\begin{tikzpicture}
  \umlinterface{ResourceStore}{}{
  + getRepresentation(ResourceIdentifier, RepresentationPreferences, ConditionValidator) : Promise<Representation>\\
  + addResource(container : ResourceIdentifier, Representation, ConditionValidator) : Promise<ResourceIdentifier>\\
  + setRepresentation(ResourceIdentifier, Representation, ConditionValidator) : Promise<void>\\
  + deleteResource(ResourceIdentifier, ConditionValidator) : Promise<void>\\
  + modifyResource(ResourceIdentifier, Patch, ConditionValidator) : Promise<void>\\
  }

  \umlinterface[right=2 of ResourceStore]{ConditionValidator}{}{
  + validate(RepresentationMetadata): boolean\\
  }
  \umldep[arg=uses,pos=0.5]{ResourceStore}{ConditionValidator}
\end{tikzpicture}

\subsection*{Description}

\begin{Note}
A~\textbf{ConditionValidator} is passed to all write methods
(and possibly also read) of \textbf{ResourceStore}.
The store is responsible for calling \verb!validate!
at the right moment,
and for aborting the modification if validation fails.
The caller is responsible for writing the validation code.

\columnbreak
(The validator argument \emph{could} be optional;
 if a~store decides to not support conditional requests,
 it must throw an error if a~validator is passed.)

\columnbreak
\null

\end{Note}

\subsection*{Analysis}

\begin{Note}
Requirements:
\begin{itemize}
  \item The caller knows how to validate request conditions,
        given metadata.
\end{itemize}

\bigskip
When a~conditional request arrives, implementers must:
\begin{enumerate}
  \item call the modification method,
        passing in the validation code
  \item have \textbf{ResourceStore} call the validator
        at the right time
\end{enumerate}

\bigskip
This comes with a~couple of caveats
(details in previous section):
\begin{itemize}
  \item It assumes that metadata is available.
  \item It assumes that retrieving metadata is sufficiently cheap.
  \item It assumes that the caller has the best knowledge for checking conditions.
  \item For every store implementation,
        it must be tested whether every method checks the conditions.
\end{itemize}

\columnbreak
\null


\columnbreak
\null

\end{Note}


\clearpage
\section*{Approach 3 to conditional requests: \emph{passing the conditions}}
\begin{tikzpicture}
  \umlinterface{ResourceStore}{}{
  + getRepresentation(ResourceIdentifier, RepresentationPreferences, Conditions) : Promise<Representation>\\
  + addResource(container : ResourceIdentifier, Representation, Conditions) : Promise<ResourceIdentifier>\\
  + setRepresentation(ResourceIdentifier, Representation, Conditions) : Promise<void>\\
  + deleteResource(ResourceIdentifier, Conditions) : Promise<void>\\
  + modifyResource(ResourceIdentifier, Patch, Conditions) : Promise<void>\\
  }

  \umlinterface[right=2 of ResourceStore]{Conditions}{
    + matchesEtag : string[]\\
    + notMatchesEtag : string[]\\
    + modifiedSince: date?\\
    + unmodifiedSince: date?\\
  }{}
  \umldep[arg=uses,pos=0.5]{ResourceStore}{Conditions}

  \umlclass[below=1 of Conditions]{ConditionValidator}{}{
  + matches(conditions : Conditions, eTag : string?, lastModified : date?): boolean\\
  }
  \umldep[arg=uses,pos=0.5]{ConditionValidator}{Conditions}
\end{tikzpicture}

\subsection*{Description}

\begin{Note}
\textbf{Conditions} are passed to all write methods
(and possibly also read) of \textbf{ResourceStore}.
The store is responsible for validating conditions
at the right moment,
and for aborting the modification if validation fails.
The caller is responsible for writing the validation code.

\columnbreak
If the store does not know how to validate conditions,
or if it does not have a~more optimal strategy,
it can simply call a~generic \textbf{ConditionValidator}
that knows how to validate conditions.
It can do this with both eTag and lastModified,
or try one of them before the other.

\columnbreak
(The conditions argument \emph{could} be optional;
 if a~store decides to not support conditional requests,
 it must throw an error if conditions are passed.)

\end{Note}

\subsection*{Analysis}

\begin{Note}
Requirements:
\begin{itemize}
  \item \emph{(none)}
\end{itemize}



\bigskip
When a~conditional request arrives, implementers must:
\begin{enumerate}
  \item call the modification method,
        passing in the validation code
  \item have \textbf{ResourceStore} check the conditions
        at the right time
\end{enumerate}

\bigskip
This comes with a~couple of caveats:
\begin{itemize}
  \item It assumes that the conditions do not change often.
  \item For every store implementation,
        it must be tested whether every method checks the conditions.
\end{itemize}

\columnbreak
\null


\columnbreak
\null

\end{Note}

\end{document}
