\documentclass[10pt]{article}
\usepackage[utf8]{inputenc}

% Page setup
\usepackage[a3paper,landscape,margin=2cm]{geometry}

% Typography
\usepackage[T1]{fontenc}
\usepackage[scaled]{berasans}
\usepackage[scaled]{beramono}
\renewcommand*\familydefault{\sfdefault}
\usepackage{microtype}
\parindent 0pt

\newcommand\component[1]{\mbox{\bf #1}}
\newcommand\field[1]{\mbox{\tt #1}}

% TikZ
\usepackage{./tikz-uml}
\usetikzlibrary{positioning}

% Headings
\makeatletter
\def\@maketitle{%
  {\LARGE\bf\@title\par}
  \vskip .5em
  {\large\@author\ -- \@date\par}
  \vskip 2em
}
\makeatother

% Notes
\usepackage{multicol}
\newenvironment{Note}
  {\begin{multicols}{3}%
     \parskip 1em}
  {\end{multicols}}

% Lists
\usepackage{enumitem}
\setlist{nosep,topsep=-1em}

% Document metadata
\title{
  Solid server -- Selected architectural diagrams v1.3.0
  \it (status: draft)
}
\author{Ruben Verborgh}

\begin{document}

\maketitle


\section*{Purpose}
This document conveys views
on important architectural considerations for a~Solid server.

It is mainly intended as a~tool for discussing,
raising questions,
and highlighting concerns.


\section*{Legend}
The architectural diagram follows standard UML notation.

For more specific symbols that are not part of UML,
Node.js/JavaScript/TypeScript conventions were used as follows:

\begin{description}
  \item[T?] represents a~value that is either not present
            or a~value of type~T.
  \item[Promise<T>] represents a~value that will asynchronously resolve
                    to a~value of type~T.
  \item[Readable<T>] represents an asynchronous one-time readable stream
                     of values of type~T.
  \item[Buffer] is an in-memory buffer of bytes,
                possibly with a~character encoding.
\end{description}

\clearpage

\section*{Overview of LDP and Access Control}

\newcommand\ResourceStoreBody{%
  + getRepresentation(ResourceIdentifier, RepresentationPreferences, Conditions?) : Promise<Representation>\\
  + addResource(container : ResourceIdentifier, Representation, Conditions?) : Promise<ResourceIdentifier>\\
  + setRepresentation(ResourceIdentifier, Representation, Conditions?) : Promise<void>\\
  + deleteResource(ResourceIdentifier, Conditions?) : Promise<void>\\
  + modifyResource(ResourceIdentifier, Patch, Conditions?) : Promise<void>\\
}

\begin{tikzpicture}

\begin{umlpackage}{Server}
  \umlinterface[]{HttpHandler}{}{
    + canHandle(HttpRequest) : Promise<boolean>\\
    + handle(HttpRequest, HttpResponse) : Promise<void>\\
  }

  \umlclass[right=1 of HttpHandler]{HttpServer}{}{
    + HttpServer(Array<HttpHandler>)\\
    + listen(port : int) : void\\
    + handle(HttpRequest, HttpResponse) : void\\
  }
  \umluniaggreg[mult=*]{HttpServer}{HttpHandler}
\end{umlpackage}

\begin{umlpackage}[y=-5]{LDP}
  \umlclass{LdpHandler}{}{
  }
  \umlimpl{LdpHandler}{HttpHandler}

  \begin{umlpackage}{HTTP}
    \umlsimpleclass[right=1 of LdpHandler.north east]{ResponseWriter}
    \umluniaggreg[mult=1,above,anchor1=10,anchor2=west,geometry=-|-,arm1=0.5]{LdpHandler}{ResponseWriter}
    \umlsimpleclass[right=2 of ResponseWriter.east,anchor=west]{HttpResponse}
    \umldep[arg=writes,pos=0.5,below]{ResponseWriter}{HttpResponse}

    \umlclass[below=1 of ResponseWriter.south east,anchor=north east]{RequestParser}{}{
    }
    \umluniaggreg[mult=1,below,anchor1=-10,anchor2=west,geometry=-|-,arm1=0.5]{LdpHandler}{RequestParser}

    \umlsimpleclass[right=2 of RequestParser.east,anchor=south west]{TargetExtractor}
    \umluniaggreg[arg2=1,above,anchor1=north east,anchor2=west,geometry=-|-]{RequestParser}{TargetExtractor}
    \umlsimpleclass[right=2 of TargetExtractor]{ResourceIdentifier}
    \umldep[arg=creates,pos=0.5]{TargetExtractor}{ResourceIdentifier}

    \umlsimpleclass[below=0.25 of TargetExtractor.south east,anchor=north east]{BodyParser}
    \umluniaggreg[mult=*,anchor1=east,anchor2=west,geometry=-|-]{RequestParser}{BodyParser}
    \umlsimpleclass[below=0.25 of ResourceIdentifier.south west,anchor=north west]{Representation}
    \umldep[arg=creates,pos=0.5]{BodyParser}{Representation}
    \umlsimpleclass[right=1 of Representation]{Patch}
    \umlinherit{Patch}{Representation}

    \umlsimpleclass[below=0.25 of BodyParser.south east,anchor=north east]{PreferenceParser}
    \umluniaggreg[mult=1,anchor1=south east,anchor2=west,geometry=-|-]{RequestParser}{PreferenceParser}
    \umlsimpleclass[below=0.25 of Representation.south west,anchor=north west]{RepresentationPreferences}
    \umldep[arg=creates,pos=0.5]{PreferenceParser}{RepresentationPreferences}
  \end{umlpackage}


  \begin{umlpackage}[x=6,y=-8]{Operations}
    \umlinterface[]{OperationHandler}{}{
      + canHandle(Operation) : Promise<boolean>\\
      + handle(Operation) : Promise<void>\\
    }
    \umluniaggreg[anchor1=310,mult=1,geometry=|-]{LdpHandler}{OperationHandler}

    \umlsimpleclass[above=1 of OperationHandler]{Operation}
    \umldep[arg=uses,pos=0.5]{OperationHandler}{Operation}

    \umlsimpleclass[right=1 of OperationHandler.north east,anchor=south west]{CompositeOperationHandler}
    \umlimpl[anchor1=177,anchor2=14]{CompositeOperationHandler}{OperationHandler}
    \umlunicompo[anchor1=182,anchor2=10]{CompositeOperationHandler}{OperationHandler}

    \umlsimpleclass[below=.5  of CompositeOperationHandler.south west,anchor=north west]{GetOperationHandler}
    \umlsimpleclass[below=.25 of GetOperationHandler.south west,anchor=north west]{PostOperationHandler}
    \umlsimpleclass[below=.25 of PostOperationHandler.south west,anchor=north west]{PutOperationHandler}
    \umlsimpleclass[below=.25 of PutOperationHandler.south west,anchor=north west]{PatchOperationHandler}
    \umlimpl[anchor1=west]{GetOperationHandler}{OperationHandler}
    \umlimpl[anchor1=west]{PostOperationHandler}{OperationHandler}
    \umlimpl[anchor1=west]{PutOperationHandler}{OperationHandler}
    \umlimpl[anchor1=west]{PatchOperationHandler}{OperationHandler}
  \end{umlpackage}

  \umldep[arg=creates,pos=0.3,geometry=|-|,arm1=-2.5]{RequestParser}{Operation}
\end{umlpackage}

\begin{umlpackage}{Storage}
  \umlinterface[below=4.25 of OperationHandler]{ResourceStore}{}{
    \ResourceStoreBody
  }
  \umluniaggreg[mult=1]{OperationHandler}{ResourceStore}
\end{umlpackage}

\begin{umlpackage}{Authentication}
  \umlinterface[left=2 of LdpHandler.west]{CredentialsExtractor}{}{
    + extractCredentials(HttpRequest) : Promise<Credentials>\\
  }
  \umluniaggreg[mult=1]{LdpHandler}{CredentialsExtractor}

  \umlsimpleclass[below=1 of CredentialsExtractor]{Credentials}
  \umldep[arg=creates,pos=0.5]{CredentialsExtractor}{Credentials}
\end{umlpackage}

\begin{umlpackage}{Authorization}
  \umlinterface[below=4 of CredentialsExtractor]{Authorizer}{}{
    + ensurePermissions(Credentials,
                        ResourceIdentifier,\\*
                        PermissionSet) : Promise<void>\\
  }
  \umldep[arg=uses,pos=0.3]{Authorizer}{Credentials}
  \umluniaggreg[anchor1=230,mult=1,geometry=|-,weight=0.6]{LdpHandler}{Authorizer}

  \umlsimpleclass[below=1 of Authorizer]{AclBasedAuthorizer}
  \umlimpl{AclBasedAuthorizer}{Authorizer}
  \umluniaggreg[mult=1,geometry=|-]{AclBasedAuthorizer}{ResourceStore}
\end{umlpackage}

\end{tikzpicture}

\clearpage

\section*{Resources and Representations}
\begin{Note}
The intention of \component{ResourceIdentifier} and \component{Representation}
is to capture the REST notion of a~resource and its representation.
In~the~case of a~photograph,
the resource is the photograph itself,
whereas a~representation is a~concrete manifestation of that photograph
with a~certain resolution and file~type.
In~the~case of an RDF document,
the~resource is the RDF graph,
and concrete representations serialize that graph
into Turtle or specific framings of~JSON-LD.
\columnbreak

For all practical purposes,
\component{ResourceIdentifier} can just be a~\component{URL};
the terminology is mainly used to emphasize
the resource/representation notion of REST.
Also, there is no \component{Resource} class,
because resources are always manipulated through representations in REST,
so we only need to \emph{identify} resources,
and only deal with them through their representations.
\columnbreak

Crucially, as the diagram below shows,
the \component{Representation} interface
can have vastly different underlying in-memory structures,
such as strings, binary streams, RDF streams, etc.
So they can be photographs as well as RDF streams,
and most other classes handling them do not need to care.
This enables back-ends to be RDF-aware when they need to,
and RDF-oblivious when they do~not.
\end{Note}

\bigskip

\begin{tikzpicture}
  \umlinterface{Representation}{
    + identifier : ResourceIdentifier?\\
    + metadata : RepresentationMetadata\\
    + data : Readable<Object>\\
    + dataType : String\\
  }{}

  \umlinterface[left=2 of Representation]{ResourceIdentifier}{
  }{}
  \umluniaggreg[mult=0..1]{Representation}{ResourceIdentifier}

  \umlinterface[below=2 of Representation]{RepresentationMetadata}{
  }{
    + raw : Array<Quad>\\
    + byteSize : int?\\
    + contentType : String?\\
    + encoding : String?\\
    + language : String?\\
    + dateTime : Date?\\
    + profiles : Array<String>\\
  }
  \umluniaggreg[mult=1]{Representation}{RepresentationMetadata}

  \umlclass[right=2 of Representation.north east]{BinaryRepresentation}{
    + data : Readable<Buffer>\\
  }{}
  \umlimpl{BinaryRepresentation}{Representation}

  \umlclass[right=2 of Representation.south east]{QuadRepresentation}{
    + data : Readable<Quad>\\
  }{}
  \umlimpl{QuadRepresentation}{Representation}

\end{tikzpicture}

\bigskip

\begin{Note}
The \field{dataType} field returns the name of the class
that elements of the \field{data} readable stream will have,
for instance,
\component{Buffer} or \component{Quad}.
\columnbreak

Based on the \field{dataType} and \field{metadata} fields,
other components can decide whether or not the representation
is acceptable to the user agent,
and, if this is not the case,
convert to a~format that is.
For instance,
a~\verb!text/turtle! stream is acceptable
for a~user agent that requested \verb!text/*!,
whereas a~\component{Readable<Quad>} will still require serialization.

\columnbreak
The \component{RepresentationMetadata} interface
essentially exposes a~set of RDF triples
that describe properties about the representation.
For convenience,
direct getters to common properties can be added,
non-binding examples of which are shown in the diagram.

\end{Note}

\clearpage

\section*{ResourceStore implementations}
\begin{tikzpicture}
  \umlinterface{ResourceStore}{}{\ResourceStoreBody}


  % FileSystemStore

  \umlclass[right=2 of ResourceStore]{FileSystemStore}{
    - ResourceMapper mapper\\
  }{}
  \umlimpl{FileSystemStore}{ResourceStore}

  \umlinterface[right=1 of FileSystemStore]{ResourceMapper}{}{
    + mapFilePathToUrl(File): Promise<\{URL, RepresentationMetadata\}>\\
    + mapUrlToFilePath(URL, RepresentationMetadata): Promise<File>\\
  }
  \umluniaggreg[mult=1]{FileSystemStore}{ResourceMapper}


  % KeyValueStore

  \umlabstract[above=1 of FileSystemStore.north west,anchor=south west]{KeyValueStore}{}{
    \# get(ResourceIdentifier) : Promise<BinaryRepresentation>\\
    \# replace(ResourceIdentifier, BinaryRepresentation) : Promise<void>\\
    \# delete(ResourceIdentifier) : Promise<void>\\
  }
  \umlimpl[anchor1=west,anchor2=north east]{KeyValueStore}{ResourceStore}

  \umlsimpleclass[right=1 of KeyValueStore.north east]{RedisStore}
  \umlinherit[anchor1=west]{RedisStore}{KeyValueStore}

  \umlsimpleclass[right=1 of KeyValueStore.east]{CassandraStore}
  \umlinherit[anchor1=west]{CassandraStore}{KeyValueStore}


  % TripleStore

  \umlclass[below=1 of FileSystemStore.south west,anchor=north west]{TripleStore}{}{
    + getRepresentation(\ldots) : Promise<QuadRepresentation>\\
  }
  \umlimpl[anchor1=west,anchor2=south east]{TripleStore}{ResourceStore}

  \umlclass[right=1 of TripleStore]{SparqlEndpointStore}{
    - endpoint: URL\\
  }{}
  \umlinherit{SparqlEndpointStore}{TripleStore}


  % RepresentationConvertor

  \umlclass[below=1 of TripleStore.south west,anchor=north]{RepresentationConvertingStore}{
    - source : ResourceStore\\
    - convertors : Array<RepresentationConvertor>\\
  }{}
  \umlimpl[anchor1=west,anchor2=-18,geometry=-|]{RepresentationConvertingStore}{ResourceStore}

  \umlinterface[right=1 of RepresentationConvertingStore]{RepresentationConvertor}{}{
    + supports(Representation, RepresentationPreferences): Promise<boolean>\\
    + convert(Representation, RepresentationPreferences): Promise<Representation>\\
  }
  \umluniaggreg[mult=*]{RepresentationConvertingStore}{RepresentationConvertor}


  % CompositeResourceStore

  \umlclass[below=1 of RepresentationConvertingStore.south west,anchor=north west]{CompositeResourceStore}{
    - sources : Map<UrlPattern, ResourceStore>\\
  }{}
  \umlimpl[anchor1=173,anchor2=-22,geometry=-|]{CompositeResourceStore}{ResourceStore}
  \umlunicompo[mult=*,anchor1=190,anchor2=-28,geometry=-|]{CompositeResourceStore}{ResourceStore}

\end{tikzpicture}

\bigskip

\begin{Note}
A~\component{ResourceStore} will \emph{try} to satisfy
any \component{RepresentationPreferences} passed to it,
but only if this is reasonably easy for the store in question.
For instance, a~SPARQL endpoint can typically
generate N-Triples as easily as Turtle,
so it makes sense to directly generate N-Triples
if the client prefers this.
On the other hand,
a~file system will typically only have one representation on disk,
so it is fine to always serve that single representation,
regardless of client preferences.
\columnbreak

Optionally, a~\component{RepresentationConvertingStore} can be used
to satisfy client preferences more accurately.
It has access to \component{RepresentationConvertor} instances,
which could (for instance) convert a~stream of quads
into Turtle or a~specific JSON-LD frame.
It can decorate any existing \component{ResourceStore}
to extend it with more kinds of representations
such as different content types.
\columnbreak

A~\component{CompositeResourceStore} can be used
to have multiple back-ends on one pod,
each answering to different URL patterns.
This mechanism \emph{could} be used
also to serve large files like images,
or static assets such as apps or scripts.
\end{Note}

\section*{ResourceStore atomicity and conditional requests}
\begin{tikzpicture}
  \umlinterface{ResourceStore}{}{\ResourceStoreBody}

  \umlclass[below=1 of ResourceStore]{Conditions}{
    + matchesEtag : string[]\\
    + notMatchesEtag : string[]\\
    + modifiedSince: date?\\
    + unmodifiedSince: date?\\
  }{
    + matches(metadata : RepresentationMetadata): boolean\\
    + matches(eTag : string?, lastModified : date?): boolean\\
  }
  \umldep[arg=uses,pos=0.5]{ResourceStore}{Conditions}

  \umlinterface[right=2 of ResourceStore.north east, anchor=south west]{AtomicResourceStore}{}{
  }
  \umlimpl{AtomicResourceStore}{ResourceStore}

  \umlclass[right=2 of ResourceStore.east]{LockingResourceStore}{
    - ResourceStore source\\
    - ResourceLocker locks\\
  }{
  }
  \umlimpl{LockingResourceStore}{ResourceStore}
  \umlimpl{LockingResourceStore}{AtomicResourceStore}
  \umlunicompo[mult=1,anchor1=202,anchor2=-6]{LockingResourceStore}{ResourceStore}

  \umlinterface[right=2 of LockingResourceStore.east]{ResourceLocker}{
  }{
    + acquire(ResourceIdentifier) : Promise<Lock>\\
  }
  \umluniaggreg[mult=1]{LockingResourceStore}{ResourceLocker}

  \umlinterface[below=1 of ResourceLocker.south]{Lock}{
  }{
    + release() : Promise<void>\\
  }
  \umldep[arg=creates,pos=0.5]{ResourceLocker}{Lock}

\end{tikzpicture}

\bigskip

\begin{Note}
The \component{ResourceStore} interface has been designed
such that each of its methods \emph{can} be implemented
in an atomic way:
for each CRUD operation,
only one dedicated method needs to be called.
A~fifth method enables the optimization of partial updates
with \verb!PATCH!.
It is up to the implementer of the interface
to (not) make an implementation atomic.
For some implementations,
such as triple stores or other database back-ends,
atomicity is a~given.
We~\emph{could} explicitly indicate atomicity
by having such implementations
implement the (otherwise empty) \component{AtomicResourceStore} interface
as a~tag.

\columnbreak

Some back-ends are not atomic by themselves,
such as a~file system,
where a~read+append sequence could unknowingly
be interrupted by a~write
that thereby breaks atomicity.
Instead of having to implement
a~dedicated locking mechanism for every non-atomic back-end,
these stores can be made atomic
by decorating them with a~\component{LockingResourceStore}.
This class wraps another \component{ResourceStore}
and adds a~locking mechanism,
of which different implementations can exist.

\columnbreak

It is important to emphasize
that atomicity is \emph{not} the only reason
for the design of the \component{ResourceStore} interface.
Another consideration is \field{modifyResource},
which allows us to optimize modifications in a~backend-specific way.
Since we expect small modifications to larger resources
to be a~common for Solid apps,
we need to be able to handle those efficiently.
\field{modifyResource} gives implementations the freedom
on how to apply patches,
such that they can pick whichever option is most efficient
for a~given patch
and, if desired, support atomicity.

\end{Note}

\begin{Note}
The \component{Conditions} class
represents the conditions of an HTTP conditional request.
It~is passed to all write methods
(and possibly also read) of \component{ResourceStore}.
The~store is responsible for validating conditions
at the right moment
and, should validation fail,
for aborting the modification by throwing an error.

\columnbreak
If the store knows how to validate conditions,
it can use the raw exposed fields on \component{Conditions}.
If it does not,
it can call \field{modifyResource}
with both ETag and the last modified date,
or try one of them before the other.
Finally,
if it knows about neither ETag nor last modified date,
it can pass the metadata as a~whole.

\columnbreak
The conditions argument is optional,
and only passed for conditional requests.
If a~store decides not to support conditional requests,
it must throw an error if~conditions are passed.

\end{Note}

\clearpage

\section*{Patch}
\begin{Note}
A~\component{Patch} contains a~description
of changes to be made to a~certain
(representation of a) resource.
The \component{Patch} object itself
does not know how to \emph{apply} this patch;
it is merely a~data object.
\columnbreak

\null
\columnbreak

\null
\end{Note}

\bigskip

\begin{tikzpicture}
  \umlinterface{Patch}{}{}

  \umlclass[right=2 of Patch.north east,anchor=south west]{LineBasedPatch}{
    + deletions : Map<int, string>\\
    + additions : Map<int, string>\\
  }{}
  \umlimpl[anchors=west and north east]{LineBasedPatch}{Patch}

  \umlclass[below=1 of LineBasedPatch.south west,anchor=north west]{GraphPatternPatch}{
    + where  : Array<QuadPattern>\\
    + delete : Array<QuadPattern>\\
    + insert : Array<QuadPattern>\\
  }{}
  \umlimpl[anchor2=east]{GraphPatternPatch}{Patch}

  \umlclass[below=1 of GraphPatternPatch.south west,anchor=north west]{BinaryPatch}{}{}
  \umlimpl[anchors=north west and south east]{BinaryPatch}{Patch}

  \umlclass[below=1 of BinaryPatch.south west,anchor=north west]{ImageFilter}{}{}
  \umlimpl[anchors=north west and south]{ImageFilter}{Patch}
\end{tikzpicture}

\begin{Note}
A~\component{ResourceStore} \emph{might} have knowledge
on how to apply certain types of patches itself.
For instance, file-based stores
might have built-in support for \component{LineBasedPatch},
and SPARQL endpoints or in-memory RDF stores
likely have built-in support for \component{GraphPatternPatch}.
\columnbreak

There is case to be made for a~\component{Patcher} interface
for objects that can apply all patches of a~certain type
to certain representations.
For instance,
a~\component{GraphPatternPatch} could be applied
to RDF graphs serialized as documents,
by a~\component{GraphPatternPatcher}
that operates independently of any specific store.

\columnbreak

\null
\end{Note}

\clearpage

\section*{Quota}
\begin{tikzpicture}
  \umlsimpleclass[]{AdministrationApi}{}{}
  \umlinterface[right=2 of AdministrationApi]{HttpHandler}{}{}
  \umlimpl{AdministrationApi}{HttpHandler}

  \umlinterface[below=2 of AdministrationApi]{SizeReporter}{}{
    + getSize(ResourceIdentifier) : Promise<Number>\\
  }
  \umldep[arg=uses,pos=0.5]{AdministrationApi}{SizeReporter}

  \umlclass[right=2 of SizeReporter]{SizeCache}{
    - source : ResourceStore\\
    - reporter : SizeReporter\\
  }{}
  \umlimpl{SizeCache}{SizeReporter}
  \umlunicompo[mult=1,anchor1=202,anchor2=-12]{SizeCache}{SizeReporter}

  \umlinterface[below=1.5 of SizeCache]{ResourceStore}{}{}
  \umlimpl{SizeCache}{ResourceStore}
  \umlunicompo[mult=1,anchor1=-124,anchor2=128]{SizeCache}{ResourceStore}

  \umlsimpleclass[below=2 of SizeReporter]{FileSystemStore}
  \umlimpl{FileSystemStore}{ResourceStore}
  \umlimpl{FileSystemStore}{SizeReporter}

\end{tikzpicture}

\begin{Note}
Storage quota can be retrieved
through the \component{AdministrationApi},
which is an independent \component{HttpHandler}
that accesses a~\component{SizeReporter}.
The \component{SizeReporter} interface
can be implemented by stores such as \component{FileSystemStore}.

\columnbreak
Since computing quota can be expensive,
a~\component{SizeCache}
could maintain quota for subpaths,
which it invalidates upon write operations.

\columnbreak
\null

\end{Note}

\clearpage

\section*{ACL caching}
\begin{tikzpicture}
  \umlclass{AclBasedAuthorizer}{
    - source : AclCache\\
  }{
    + ensurePermissions(Credentials,
                        ResourceIdentifier,\\*
                        PermissionSet) : Promise<void>\\
  }

  \umlclass[right=2 of AclBasedAuthorizer]{AclCache}{
    - source : ResourceStore\\
  }{}
  \umldep[arg=uses,pos=0.5]{AclBasedAuthorizer}{AclCache}

  \umlinterface[below=1.5 of AclCache]{ResourceStore}{}{}
  \umlimpl{AclCache}{ResourceStore}
  \umlunicompo[mult=1,anchor1=-130,anchor2=128]{AclCache}{ResourceStore}

\end{tikzpicture}

\begin{Note}
Since ACLs will be used frequently,
we need a~mechanism for caching them.
Importantly, we need a~way to \emph{invalidate} the cache
every time a~write operation happens
to ACLs that can affect a~given document.

\columnbreak
To this end,
the \component{AclCache} will wrap around a~\component{ResourceStore}
and intercept all write requests,
such that it can invalidate parts of its cache
when writes to ACL~documents arrive.

\columnbreak
\null

\end{Note}

\end{document}
